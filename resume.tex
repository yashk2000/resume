\documentclass[letterpaper,11pt]{article}

\usepackage{latexsym}
\usepackage[empty]{fullpage}
\usepackage{titlesec}
\usepackage{marvosym}
\usepackage[usenames,dvipsnames]{color}
\usepackage{verbatim}
\usepackage{enumitem}
\usepackage[pdftex, hidelinks]{hyperref}
\usepackage{fancyhdr}

\usepackage[charter]{mathdesign} % Bitstream Charter
% \usepackage{newpxtext,newpxmath} % Palatino

\pagestyle{fancy}
\fancyhf{} % clear all header and footer fields
\fancyfoot{}
\renewcommand{\headrulewidth}{0pt}
\renewcommand{\footrulewidth}{0pt}

% Adjust margins
\addtolength{\oddsidemargin}{-0.50in}
\addtolength{\evensidemargin}{-0.50in}
\addtolength{\textwidth}{1in}
\addtolength{\topmargin}{-0.8in}
\addtolength{\textheight}{1.0in}

\urlstyle{same}

\raggedbottom
\raggedright
\setlength{\tabcolsep}{0in}

% Sections formatting
\titleformat{\section}{
  \vspace{-6pt}\scshape\raggedright\large
}{}{0em}{}[\color{black}\titlerule \vspace{-5pt}]

%-------------------------
% Custom commands
\newcommand{\resumeItem}[2]{
  \item\small{
    \textbf{#1}{: #2 \vspace{-2pt}}
  }
}

\newcommand{\resumeItemNoBullet}[2]{
  \item[]\small{
    \hspace{2pt}\textbf{#1}{: #2 \vspace{-6pt}}
  }
}

\newcommand{\resumeSubheading}[4]{
  \vspace{-2pt}\item[]
  \begin{tabular*}{0.98\textwidth}{l@{\extracolsep{\fill}}r}
      \hspace{-10pt}\textbf{#1} & #2 \\
      \hspace{-10pt}\textit{\small#3} & \textit{\small #4} \\
    \end{tabular*}\vspace{-5pt}
}

\newcommand{\resumeSubItem}[2]{\resumeItem{#1}{#2}\vspace{-1pt}}

\renewcommand{\labelitemii}{$\circ$}

\newcommand{\resumeSubHeadingListStart}{\begin{itemize}[leftmargin=*]}
\newcommand{\resumeSubHeadingListEnd}{\end{itemize}}
\newcommand{\resumeItemListStart}{\begin{itemize}}
\newcommand{\resumeItemListEnd}{\end{itemize}\vspace{-5pt}}

% custom commands
\newcommand{\shorterSection}[1]{\vspace{-12pt}\section{#1}}

%-------------------------------------------
%%%%%%  CV STARTS HERE  %%%%%%%%%%%%%%%%%%%%%%%%%%%%


\begin{document}

%----------HEADING-----------------
\begin{center}
  \small \textbf{\huge Yash Khare} \\ 
  {(+91) 6397812260} | \href{mailto:yashsja@gmail.com}{yashsja@gmail.com} | \href{https://yashk2000.github.io}{yashk2000.github.io} \\
  \href{https://www.linkedin.com/in/yashk2000/}{linkedin.com/in/yashk2000} | \href{https://github.com/yashk2000}{github.com/yashk2000} | \href{https://twitter.com/_p0lar_bear}{twitter.com/\_p0lar\_bear} \\

 
\end{center}


%-----------SUMMARY-----------------
\shorterSection{Summary}
\small A proactive and fast learning individual seeking an opportunity to work as a dynamic software engineer utilizing my analytical \& methodical skills and relevant expertise to help the company achieve business goals while sticking to vision, mission and values.
\vspace{3pt}

%-----------EDUCATION-----------------
\shorterSection{Education}
 \resumeSubHeadingListStart
  
    \resumeSubheading
      {Amrita Vishwa Vidyapeetham}{Amritapuri, Kollam, India.}
      {Bachelors of Technology in Computer Science;  GPA: 9.18/10.0}{Expected May 2022}{}
                  \vspace{-4pt}


    \resumeSubheading
      {St.Joseph's Academy}{Dehradun, Uttarakhand, India.}
      {Higher Secondary;  Marks: 94.25\%}{May 2018}{}
                  \vspace{-4pt}

      
    \resumeSubheading
      {St.Joseph's Academy}{Dehradun, Uttarakhand, India.}
      {Secondary; Marks: 92.2\%}{May 2016}{}

  \resumeSubHeadingListEnd
  \vspace{-6pt}

  
  
%-----------EXPERIENCE-----------------
\shorterSection{Experience}
  \resumeSubHeadingListStart
      \resumeSubheading
      {MLH - Major League Hacking}{}
      {Fellow}{September 2020 - Present}
      \vspace{-5pt}
      
       I am working on making projects and experimenting with new technologies by collaborating on a series of hackathon sprints.
       \vspace{-4pt}
      
      \resumeSubheading
      {Google Summer of Code}{}
      {Student Developer}{May 2020 - September 2020}
      \vspace{-5pt}
      
       I was selected for GSoC'2020 to work on the Computer Vision Based PPI Tool 2.0 under the Mifos Initiative to train and leverge power of cloud hosted and tflite models in a Kotlin based Android application for automcatically fill PPI Surveys. 
       \vspace{-4pt}
       
    \resumeSubheading
      {Defence Research and Development Organization(DRDO)}{}
      {Intern}{November 2019 - December 2019}
      \vspace{-5pt}
      
       I worked on automatic target detection and developed and algorithm for automatic detection of moving ground targets in image sequences captured by an infrared imaging system.
       \vspace{-2pt}
       
       \resumeSubheading
      {GitHub}{}
      {GitHub Campus Expert}{August 2020 - Present}
      \vspace{-5pt}
      
       As GitHub Campus Expert, I receive training and mentorship from GitHub employees and support to help in the growth the developer community on my campus.
       \vspace{-2pt}

    \resumeSubheading
      {FOSSASIA}{}{Intern}{May 2019 - August 2019}
            \vspace{-5pt}
            
      I overhauled cloud deployment of 2 applications, resulting in reduced run time performance by 30\%. I also worked on making Badge Magic, the simulation of a hardware LED name badge and Phimp.me, a photo editing tool. 
      
  \resumeSubHeadingListEnd
    \vspace{-4pt}



%-----------ACADEMIC PROJECTS AND INTERNSHIPS-----------------
\shorterSection{Projects}
  \resumeSubHeadingListStart
  
   \resumeSubItem{Vision PPI}{I worked on Vision PPI as a part of Google Summer of Code`20. I trained models for image labelling using Tensorflow and converted them into a TensorflowLite model to be deployed on an Android app built using Java and Kotlin.}
	            \vspace{-4pt}
	            
     \resumeSubItem{Psychic CCTV}{This is a video analysis tool, built with PyTorch with a GUI using pyqt, for low resolution CCTV footages which uses YOLO, super resolution and sound track separation to automatically detect points of interest in a video.}
	 \vspace{-4pt}

    \resumeSubItem{Ocellus}{Ocellus is an OSINT Data Analysis web platform that fetches information related to ip/mac address, email IDs, social media accounts, etc and analyses the information. The platform also provides a feature for analysing malware in apk files.}   
   \vspace{-4pt}
   
  \resumeSubItem{Kiwix}{Kiwix is an offline reader for Web content, made for android platforms using Kotlin. One of its main purposes is to make Wikipedia available offline. I am one of the top contributor of the project with 100+ contributions.}
              \vspace{-4pt}

  
  \resumeSubItem{amFOSS CMS}{This is a flutter application using which club members can login into the amFOSS Club Management System and view club related details fetched using the graphQL CMS APIs, also made by amFOSS members.}   
              \vspace{-4pt}

      
    \resumeSubItem{Tweegenous}{The tool was designed for people who speak indigenous languages. It collects tweets related to natural disaster and translates them in the language desired by the user and alerts people instantly if there is a natural calamity or any disaster headed their way by translating tweets. It is a two way system, for both the authorities and people.}   
                \vspace{-4pt}
    
  \resumeSubHeadingListEnd
    \vspace{1pt}



%-----------SKILLS-----------------
\shorterSection{Skills}
  \resumeSubHeadingListStart
  \small
    \item{\textbf{Languages}{: Python, Java, Kotlin, Dart, C++, C, Bash.}}
    \vspace{-5pt}
    \item{\textbf{Skills}{: Android Development, Flutter, Computer Vision, git, Keras, PyTorch, openCV, PySimpleGUI}}
\resumeSubHeadingListEnd
  \vspace{-5pt}





%-----------Addtional Experience & Achievements-----------------
\shorterSection{Additional Experience \& Achievements}
  \resumeSubHeadingListStart
  \small
  \item{Was selected for HackMIT 2020, the annual hackathon held by Massachusetts Institute of Technology.}
    \vspace{-6pt}
  \item{Was selected for attending the AI Summer School 2020, held by AI Singapore.}
    \vspace{-6pt}
    \item{Got invited to FOSSASIA OpenTech Summit 2020, Singapore, to give a talk on The Optimal Pathway to Deep Learning}
    \vspace{-6pt}
    \item{Was selected as a Google Code-In 2019 mentor for the Wikimedia Foundation and FOSSASIA.}
    \vspace{-6pt}
    \item{Got selected for Hack The North 2019, Canada`s biggest Hackathon, held at the University of Waterloo(with travel funding)}
    \vspace{-6pt}
    \item{My paper on Infrared Image Enhancement got selected to be presented in the International Conference on Optics and Electro-Optics 2019 held at Instruments Research and Development Establishment, a DRDO establishment}
    \vspace{-6pt}
    \item{Was among the 90 students who were selected for attending the Undergraduate Summer School 2019, held by IISc Bangalore}
    \vspace{-6pt}
    \item{Won 2nd prize in IBM-Cloud Category in FOSSASIA UNESCO Hackathon held in Singapore.}
  \resumeSubHeadingListEnd
%-------------------------------------------
\end{document}
